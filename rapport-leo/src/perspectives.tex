\section{Perspectives}

  \subsection{Autres algorithmes}

    Seul le premier algorithme a été traité, il est celui qui couvre le plus de cas parmi les formations. Le deuxième, avec taux de boursiers et taux de résidents, est en réalité très similaire, il a été partiellement implémenté et diverses preuves ont été débutées. Il est à noter que les preuves du premier algorithme ont été réalisées de telle sorte qu'elles devraient facilement s'adapter au deuxième algorithme. Le troisième algorithme, qui gère les internats, est différent et bien plus compliqué, mais il ne concerne qu'un petit nombre de formations.

  \subsection{Implémentation originale}

    On souhaite pouvoir vérifier l'implémentation originale en Java. Pour cela, un outils appelé \texttt{jml2why3} est développé par Benedikt \bsc{Becker}. Il devrait permettre de produire du code WhyML à partir du code Java original. On espère alors pouvoir transposer les preuves de notre implémentation WhyML au code WhyML généré.
