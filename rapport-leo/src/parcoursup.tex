\section{Parcoursup}
  \subsection{Généralités}
    Parcoursup est une plateforme gouvernementale française utilisée par les étudiants pour candidater à des formations de l'enseignement supérieur. Le principe général est le suivant:
    \begin{itemize}
      \item les étudiants candidatent à plusieurs formations ;
      \item chaque formation classe l'ensemble des étudiants selon les critères de son choix\footnote{On parlera de \emph{classement pédagogique}.} ;
      \item la plateforme Parcoursup apporte des modifications aux classements pédagogiques\footnote{Le nouveau classement est appelé l'\emph{ordre d'appel}.}, selon des critères et des algorithmes publics et notifie les candidats acceptés, qui doivent alors faire un choix parmi les formations où ils sont acceptés\footnote{Tout en gardant la possibilité de choisir une autre formation où ils sont sur liste d'attente pour le moment.} ;
      \item lorsqu'un candidat choisit une formation parmi celles où il a été accepté, cela libère des places dans les autres, qui sont alors proposées aux candidats suivants sur l'ordre d'appel.
    \end{itemize}

    Plus de détails sont donnés dans~\cite{AlgoPS}.

  \subsection{Algorithme avec taux de boursiers}

    Pour passer du classement pédagogique à l'ordre d'appel, il existe plusieurs cas, selon les formations. Chaque cas correspond à un algorithme particulier. Le cas le plus simple est celui avec un taux de boursiers.

    Pour chaque formation sélective, le recteur de l'académie fixe un taux minimum de boursiers à qui l'on doit proposer une place dans la formation.

    Initialement, l'ordre d'appel est vide. On dit que le taux est contraignant s'il reste au moins un boursier qui n'a pas été ajouté à l'ordre d'appel et si le fait d'ajouter un candidat non-boursier à l'ordre d'appel rendrait le taux de boursiers dans l'ordre d'appel inférieur au taux minimum.

    Pour remplir l'ordre d'appel, on procède comme suit:

    \begin{itemize}
      \item si le taux n'est pas contraignant, alors, on prend le meilleur candidat, qu'il soit boursier ou non ;
      \item si le taux est contraignant, alors, on prend le meilleur candidat boursier.
    \end{itemize}

  \subsection{Propriétés de l'algorithme avec taux de boursiers}

    On souhaite vérifier cinq propriétés de l'algorithme. Elles ont été publiées avec la description des différents algorithmes\cite{AlgoPS}. On a introduit une propriété $P_0$, qui est une propriété implicite du document.

    \begin{description}
      \item[$P_0$] L'ordre d'appel est une permutation du classement pédagogique. Ce qui signifie que l'on n'a pas ajouté ou perdu de candidat en modifiant le classement.
      \item[$P_1$] Pour tout préfixe de l'ordre d'appel, soit le taux de boursiers est respecté, soit il n'y a plus aucun boursier après. Ce qui signifie que l'on a toujours respecté le taux si c'était possible.
      \item[$P_2$] Un candidat boursier avec le rang $r$ dans le classement pédagogique aura au moins le rang $r$ dans l'ordre d'appel. Ce qui signifie qu'un boursier n'est jamais doublé par personne, et donc, que son rang ne peut que devenir meilleur ou rester le même.
      \item[$P_3$] Un candidat non-boursier avec le rang $r$ dans le classement pédagogique aura au mieux le rang $r$ dans l'ordre d'appel et au pire le rang $r \times (1 + q_b / (100 - q_b)$ avec $q_b$ le taux de boursiers. Ce qui signifie qu'un candidat non-boursier ne double jamais personne, et donc, que son rang ne peut que rester le même ou être moins bon ; cela signifie aussi qu'il ne peut pas perdre plus de places que ce qu'impose le taux minimum de boursiers.
      \item[$P_4$] Parmi toutes les permutations possibles du classement pédagogique qui respectent $P_1$, l'ordre d'appel est celle qui minimise le nombre d'inversions --- distance de Kendall-tau. Cela signifie que pour faire respecter le taux de boursiers, on a modifié le classement pédagogique le moins possible.
      \item[$P_5$] L'ordre d'appel est le minimum selon l'ordre lexicographique induit par les classements parmi toutes les permutations possibles du classement pédagogique qui respectent $P_1$. Cela signifie que pour faire respecter le taux de boursiers, on a favorisé les meilleurs candidats autant que faire se peut.
    \end{description}
