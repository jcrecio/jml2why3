\documentclass{article}

\usepackage[utf8]{inputenc}

\usepackage{amsthm}
\usepackage{amsfonts}
\usepackage{amsmath}

\newtheorem{thm}{Theorem}
\newtheorem{lemma}{Lemma}
\newtheorem{corollary}{Corollary}
\newcommand{\capa}{\hbox{cap}}
\newcommand{\rank}{\hbox{rank}}
\newcommand{\NN}{\mathbb{N}}

\DeclareMathOperator{\rang}{rang}
\DeclareMathOperator{\Inversions}{Inversions}
\DeclareMathOperator{\NB}{NB}
\DeclareMathOperator{\CB}{CB}

\begin{document}
\title{Algorithmes de calcul des ordres d'appel}
\author{Hugo Gimbert}
\maketitle


\section{Cas d'un seul taux boursier}

Soit $C$ l'ensemble des candidats partitionné en deux sous-ensemble
$B$ les candidats boursiers et $\NB$ les candidats non-boursiers.

L'algorithme du calcul de l'ordre d'appel avec un seul taux
prend en entrée un taux minimum de candidats boursiers
$q_B\in [0,1]$
et
un classement pédagogique $\rang:C \to 1\ldots |C|$
des candidats.
On suppose que les rangs dans le classement pédagogique
sont des entiers consécutifs à partir de $1$
sans ex-aequo, c'est-à-dire que $\rang(C)=\{1\ldots |C|\}$.
Ce classement induit une permutation $(c_i)_{i\in 1\ldots |C|}$,
c'est l'unique permutation telle que $\forall i\in 1\ldots |C|, \rang(c_i) = i$.


\subsection{Spécification}


L'algorithme calcule  une autre permutation
$(d_k)_{k\in 1\ldots |C|}$ des candidats, appelée l'ordre d'appel,
qui garantit les propriétés suivantes.
\begin{itemize}
 \item[(P1)]
 Pour tout $k$, l'ensemble de candidats $d_1,\ldots,d_k$ contient au moins $q_B  * k$  candidats boursiers ; ou sinon, aucun candidat parmi $d_{k+1}\ldots d_{|C|}$ n'est boursier. 
 \item[(P2)]
Un candidat boursier qui a le rang $r$ dans le classement pédagogique n’est jamais doublé par personne et aura donc un rang inférieur ou égal à $r$ dans l’ordre d’appel.
\item[(P3)]
Un candidat non boursier qui a le rang $r$ dans le classement pédagogique ne double jamais personne et aura un rang compris entre $r$ et l'arrondi à l'entier supérieur de $r(1 + q_B / (1 - q_B) )$ dans l’ordre d’appel. 
\item[(P4)]
Comparé au classement pédagogique, l’ordre d’appel minimise le nombre d’inversions (distance de Kendall-tau), parmi ceux qui garantissent la première propriété. 
\item[(P5)]
Si l’on munit l’ensemble des sélections ordonnées de candidats de l’ordre lexicographique induit par les classements, alors l’ordre d’appel est le maximum parmi toutes les sélections qui garantissent la première propriété.
\end{itemize}

\subsection{Implémentation}

L'algorithme commence par extraire de  $(c_i)_{i\in 1\ldots |C|}$ les
permutations correspondantes $(b_i)_{i\in 1\ldots |B|}$ des boursiers
et $(nb_i)_{i\in 1\ldots |\NB|}$ des non-boursiers.
Elles sont caractérisées par
\begin{itemize}
\item
$(b_i)_{i\in 1\ldots |B|}$ est une permutation de $B$,
\item
$\forall 1 \leq i \leq j \leq |B|, \rang(b_i) \leq \rang(b_j)$.
\item
$(nb_i)_{i\in 1\ldots |\NB|}$ est une permutation de $\NB$,
\item
$\forall 1 \leq i \leq j \leq |\NB|, \rang(nb_i) \leq \rang(nb_j)$.
\end{itemize}

Le calcul est effectué en $n+1$
 étapes $0,1,\ldots ,n$.
 Au début de l'étape $k$, 
 la permutation $d_1\ldots d_{k}$
 est déjà calculée, ainsi que deux indices $r_k(B) \in 0\ldots |B|$
 et $r_k(\NB) \in 0\ldots |\NB|$.
 Initialement $k=0$ et $r_0(B)=0$ et $r_0(\NB)=0$.
 Le calcul se termine au début de l'étape $k=n$.
 
 Les deux indices $r_k(B)$ et $r_k(\NB)$ sont des pointeurs dans les
 permutations $(b_i)_{i\in 1\ldots |B|}$
 et $(nb_i)_{i\in 1\ldots |\NB|}$
 qui sont mis-à-jour de manière à garantir les invariants
 \begin{align*}
&B\cap \{d_1,\ldots,d_k\}=\{b_1,\ldots, b_{r_k(B)}\}\\
&\NB\cap \{d_1,\ldots,d_k\}=\{nb_1,\ldots, nb_{r_k(\NB)}\}\enspace.
 \end{align*}
 
\medskip

Le choix de $d_{k+1}$ est fait à l'étape $k$ comme suit.

On dit que
le taux effectif de boursier est \emph{contraignant}
après $d_1\ldots d_{k}$, si ajouter
un non-boursier à la permutation ferait passer le taux de boursiers
en dessous du  minimum fixé alors qu'il reste un boursier
à sélectionner c'est-à-dire si 
\[
(r_k(B) < |B|) \text{ et } (r_k(B) < q_B * (k+1))\enspace.
\]

Le choix de $d_{k+1}$ est fait parmi un ou deux candidats \emph{éligibles}.
L'ensemble des candidats éligibles est calculé comme suit.
Si il reste un boursier non-sélectionné, i.e. si $r_k(B) < |B|$, alors le candidat $b_{r_k(B) + 1}$ est éligible.
De plus, si le taux n'est pas contraignant après $d_1\ldots d_{k}$
et qu'il reste un non-boursier non-sélectionné, i.e. si $r_k(\NB) < |\NB|$, alors le candidat $nb_{r_k(\NB) + 1}$ est éligible.

On choisit pour $d_{k+1}$ le candidat le mieux classé pédagogiquement parmi les candidats
éligibles.
 \begin{itemize}
 \item
Si c'est le boursier $d_{k+1}=b_{r_k(B) + 1}$ alors
$r_{k+1}(B)=r_k(B) + 1$ et $r_{k+1}(\NB)=r_k(\NB)$.
 \item
Si c'est le non-boursier $d_{k+1}=nb_{r_k(\NB) + 1}$ alors
$r_{k+1}(B)=r_k(B) $ et $r_{k+1}(\NB)=r_k(\NB) + 1$.
\end{itemize}
 
  
 
\subsection{Preuve des invariants}

Pour démontrer les propriétés (P1)-(P5),
il suffit de montrer que l'algorithme maintient les invariants suivants.

Pour tout $k\in 0\ldots |C|$ on note
\[
g(k)
=
\mid B \cap \{d_1,\ldots , d_k\}\mid\enspace.
\]
Donc le taux est contraignant après $d_1,\ldots, d_k$
si 
\[
( g(k) < |B|) \text{ et } (r_k(B) < q_B * (k+1))\enspace.
\]

L'algorithme satisfait les invariants suivants.
 \begin{itemize}
 \item[(I0)]
$\forall 1 \leq i \leq k$,   $g(i) = r_i(B)$.
\item[(I0')]
 $\forall 1 \leq i \leq k$,   si le taux n'est pas contraignant 
 après $d_1,\ldots, d_{i-1}$
 alors $d_i$ est le candidat le mieux classé parmi
 $C \setminus \{d_1\ldots d_{i-1}\}$.
 Si le taux est  contraignant 
  après $d_1,\ldots, d_{i-1}$
 alors $d_i$ est le candidat le mieux classé parmi
 $B \setminus \{b_1\ldots b_{g(i-1)}\}$
 \item[(I1)]
$d_1\ldots d_{k}$ est une permutation de $\{b_1\ldots b_{r_k(B)}\} \cup \{nb_1\ldots nb_{r_k(\NB)}\}$,

 \item[(I2)]
 $\forall 1 \leq i \leq k$,
 si $d_i$ et $d_k$ sont de même type alors $\rang(d_i) \leq \rang(d_k)$.
 \item[(I3)]
 $\forall 1 \leq i \leq k$,
 si $d_k$ est boursier alors $\rang(d_i) \leq \rang(d_k) \leq k$.
 \item[(I4)]
 $\forall 1 \leq i \leq k$,
 si $d_k$ est non-boursier  et $\rang(d_i) > \rang(d_k)$
 alors $d_i$ est boursier et le taux est contraignant 
  après $d_1,\ldots, d_{i-1}$.
 \item[(I5)]
 $\forall 1 \leq i \leq k$,
 si $d_i$ est non-boursier et $x_i=\mid \{ j \in 1\ldots i \mid \rang(d_j) > \rang(d_i)\} \mid $
 alors $x_i < q_B * i + 1$.
 \item[(I6)]
 $\forall 1 \leq i \leq k$,
 si le taux est contraignant  après $d_1,\ldots, d_{i}$
 alors $i < |C|$ et $d_{i+1} \in B$.
 \item[(I7)]
 $\forall 1 \leq i \leq k$,
 si  $g(i)  < q_B * i$ alors $g(i)=|B|$.
 \end{itemize}


\subsection{Preuve des propriétés à partir des invariants}

 \medskip
 
\paragraph{Propriété $(P1)$.}
 Soit $k\in 1 \ldots |C|$ tel que $g(k) < q_B * k$.
 Alors $g(k)=|B|$ d'après (I7) .
 Donc d'après (I1),
 \[
 \forall k \leq i \leq |C|, B =  \{d_1\ldots d_{i}\}\cap B\enspace.
 \]
 donc par comptage, et puisque $(d_k)_{k \in 1 \ldots |C|}$ est une permutation,
 \[
 \forall k < i \leq |C|, d_i \not \in B\enspace.
 \] 
 
\paragraph{Propriété $(P2)$.}
  Immédiate par (I3).
 
 \paragraph{Propriété $(P3)$.}
  D'après (I5) $x_k < q_B * k + 1$.
 D'après (I2) et (I3), $1\ldots \rang(d_k) \subseteq \rang(\{d_1\ldots d_k\})$.
 Donc $k=\rang(d_k) + x_k$.
 Finalement
 $k < \rang(d_k) + q_B * k + 1$
 donc
 $\rang(d_k) \leq k < 1 +  \rang(d_k) / (1 - q_B)$,
 i.e. (P3). 
 
 
 \paragraph{Propriété $(P5)$.}
 Soit $(d'_k)_{k\in 1 \ldots |C|}$ une autre permutation de $C$ différente de $(d_k)_{k\in 1 \ldots |C|}$
 et qui respecte la propriété $(P1)$.
 Soit $k$ le plus petit indice où les deux permutations diffèrent.
 On veut montrer que $\rang(d_k) < \rang(d'_k)$.

Il y a deux cas, selon que le taux est contraignant ou non
 après $d_1,\ldots, d_{k-1}$.

 Si le taux n'est pas contraignant après $d_1,\ldots, d_{k-1}$,
 alors d'après (I0'),
 $d_k$ est le candidat le mieux classé parmi
 $C \setminus \{d_1\ldots d_{k-1}\}$
 et puisque $\{d_1\ldots d_{k-1}\}=\{d'_1\ldots d'_{k-1}\}$ alors 
 $d'_k \in C \setminus \{d_1\ldots d_{k-1}\}$ donc
 en particulier $d_k$ est mieux classé que $d'_k$ i.e. $\rang(d_k) < \rang(d'_k)$.

 Si le taux est contraignant après $d_1,\ldots, d_{k-1}$
 alors d'après (I0'),
 $d_k$ est le candidat le mieux classé parmi
 $B \setminus \{b_1\ldots b_{g(k-1)}\}$.
 De plus $d'_k$ est nécessairement boursier 
 car le taux est contraignant
 et
 $(d'_k)_{k\in 1 \ldots |C|}$ respecte $(P1)$ donc
 puisque $\{d_1\ldots d_{k-1}\}=\{d'_1\ldots d'_{k-1}\}$ alors 
 $d'_k \in B \setminus \{b_1\ldots b_{g(k-1)}\}$ et
 en particulier $d_k$ est mieux classé que $d'_k$ i.e. $\rang(d_k) < \rang(d'_k)$.

 

 \medskip
 
 \paragraph{Propriété $(P4)$.}
 On utilise
 le \emph{nombre d'inversions} ou indice de Kendall-Tau
 défini pour toute permutation $(e_k)_{k\in 1 \ldots |C|}$
 des candidats par
\begin{multline*}
\Inversions((e_k)_{k\in 1 \ldots |C|}) = \mid \{ i \in 1 \ldots |C|, j \in 1 \ldots |C|,\\ (i < j \land \rang(e_i) > \rang(e_j)) \lor (i > j \land \rang(e_i) < \rang(e_j))\}\mid \enspace
\end{multline*}

Soit $(d''_k)_{k\in 1 \ldots |C|}$ qui minimise 
le nombre d'inversions
 parmi toutes les permutations qui respectent (P1).

On veut montrer que $(d_k)_{k\in 1 \ldots |C|}= (d''_k)_{k\in 1 \ldots |C|}$.
A contrario, supposons que ce soit faux,
et soit $k$ le plus petit indice où les deux permutations diffèrent:
\begin{align*}
&d_1,\ldots , d_{k-1}=d'_1,\ldots , d''_{k-1}\\
& d_k \neq d''_{k}\enspace.
\end{align*}


Montrons que 
$(d''_k)_{k\in 1 \ldots |C|}$ vérifie la propriété (I2) avec $k=|C|$.
Remarquons qu'échanger deux boursiers ou deux non-boursiers préserve
$(P1)$.
Donc si (I2) n'était pas satisfaite on pourrait échanger
les deux candidats correspondants et diminuer ainsi le nombre d'inversions,
contredisant de ce fait la minimalité de $(d''_k)_{k\in 1 \ldots |C|}$.

Et 
$(d''_k)_{k\in 1 \ldots |C|}$ vérifie la propriété (I3) avec $k=|C|$,
par le même genre de raisonnement.

\medskip

Posons $d=d_k$ et $d''=d''_k$.
Puisque les $(d_i)_{i\in 1\ldots |C|}$ sont deux à deux distincts,
que $d_1,\ldots , d_{k-1}=d'_1,\ldots , d''_{k-1}$ et que $d_k = d \neq d'' =d''_k$
alors il existe $k_0 \in k+1 \ldots |C|$ et $k_0''\in k+1 \ldots |C|$ tel que
\begin{align*}
&d = d_k = d''_{k''_0}\\
&d'' = d''_k = d_{k_0}\enspace.
\end{align*}

Il y a trois cas selon les caractéristiques des candidats $d$ et $d''$.
\begin{itemize}
\item
Supposons $d$ et $d''$  de même type. D'après (I2), 
$d$ est le mieux classé des candidats de sa catégorie dans
$|C|\setminus \{d_1,\ldots, d_{k-1}\}$.
Et $d's$ est le mieux classé des candidats de sa catégorie dans
$|C|\setminus \{d''_1,\ldots, d''_{k-1}\} = |C|\setminus \{d_1,\ldots, d_{k-1}\} $.
Donc $d=d''$.
\item

Supposons $d$  boursier. 
%Remarquons que ni $d$ ni $d'$ n'apparaissent dans
% $d_1,\ldots , d_{k-1}=d'_1,\ldots , d'_{k-1}$.
% 
Puisque $(d_i)_{i\in 1\ldots |C|}$ et $(d''_i)_{i\in 1\ldots |C|}$ 
respectent toutes deux la contrainte $P1$
et puisqu'elles coincident sur leurs prefix de longueur $k-1$
alors d'après (P5) $\rang(d_k) < \rang(d''_k)$.
Donc $\rang(d''_{k''_0}) = \rang(d_k) < \rang(d_{k_0}) = \rang(d''_k)$.
C'est une contradiction avec (I3) car $k < k''_0$ et $d = d''_{k''_0}$ est boursier.

\item

Reste le cas où $d$ n'est pas boursier et $d''$ est boursier.
Que se passerait t'il si l'on échangeait le boursier $d''=d''_k$
avec le non-boursier $d=d''_{k_0}$ et  dans $(d''_k)_{k\in 1 \ldots |C|}$?
?
Notons $(d'''_i)_{i\in 1 \ldots |C|}$ cette nouvelle permutation.

Déjà, cela diminuerait le nombre d'inversions,
i.e.
\[
\Inversions((d'''_i)_{i\in 1 \ldots |C|}) < \Inversions((d''_i)_{i\in 1 \ldots |C|})\enspace.
\]
en effet, $\rang(d''_{k_0}) = \rang(d_k) < \rang(d_{k_0}) = \rang(d''_k)$
où l'inégalité centrale est obtenue en appliquant (I3)
à $(d_i)_{i\in 1 \ldots |C|}$.

On montre maintenant que la nouvelle permutation $(d'''_i)_{k\in 1 \ldots |C|}$ respecterait $(P1)$,
ce qui contredira la minimalité de $(d''_i)_{k\in 1 \ldots |C|}$ et terminera la preuve.

Pour toute permutation des candidats $(e_i)_{i\in 1\ldots |C|}$
et tout $\ell \in 1 \ldots |C|$, notons $CB((e_i)_{i\in 1\ldots |C|},\ell)$
la propriété
\begin{itemize}
 \item[]  l'ensemble de candidats $e_1,\ldots,e_i$ contient au moins $q_B  * i$  candidats boursiers ; ou sinon, aucun candidat parmi $e_{i+1}\ldots d_{|C|}$ n'est boursier.
\end{itemize}
Ce qui fait que pour montrer que $(d'''_i)_{k\in 1 \ldots |C|}$ satisfait $(P1)$
il suffit de montrer $CB((d'''_i)_{i\in 1\ldots |C|},\ell)$ pour tout $\ell \in 1 \ldots |C|$.

\medskip

Le cas $\ell\in 1\ldots k-1$ est trivial car  $(d_i)_{k\in 1 \ldots |C|}$ satisfait $(P1)$
et $d_1\ldots d_{k-1} = d''_1\ldots d''_{k-1}$.

\medskip

Passons au cas $\ell = k$.
Prouvons tout d'abord que 
l'ensemble de candidats $d_1 \ldots d_{k-1}$ contient au moins $q_B  * (k+1)$  candidats boursiers.
En effet $(d_i)_{i\in 1\ldots |C|}$ satisfait (P1) et que
$d_{k_0}$ est boursier donc
l'ensemble de candidats $d_1,\ldots,d_{k}$ contient au moins $q_B  * (k+1)$  candidats boursiers
et on conclut puisque $d=d_k$ est non-boursier.
Puisque $d_1 \ldots d_{k-1}=d'''_1\ldots d'''_{k-1}$ alors
l'ensemble de candidats $d'''_1,\ldots,d'''_{k}$ contient au moins $q_B  * (k+1)$  candidats boursiers
donc $CB((d'''_i)_{i\in 1\ldots |C|},\ell)$ est bien vérifiée.

\medskip

Passons maintenant au cas $\ell \in k+1\ldots k_0-1$.
Pour cela montrons au préalable qu'il n'y a que des boursiers dans la suite
$d''_k,\ldots,d''_{k''_0-1}$.
Soit $\ell \in k\ldots k''_0-1$ et supposons a contrario que $d''_\ell$ 
est non-boursier.
D'après (I2), appliqué à $(d_i)_{i\in 1\ldots |C|}$
le candidat $d$ est le non-boursier le mieux classé 
parmi $C \setminus \{d_1\ldots d_{k-1}\}$.
Et par (I2) appliquée à $(d''_i)_{i\in 1\ldots |C|}$,
le candidat $d''_\ell$ est le non-boursier le mieux classé
parmi $C \setminus \{d''_1\ldots d''_{\ell-1}\}$.
Or $\{d_1\ldots d_{k-1}\}\subseteq \{d''_1\ldots d''_{\ell-1}\}$
donc $\rang(d)\leq \rang(d''_\ell)$.
Donc $\rang(d''_{k''_0}) \leq \rang(d''_\ell)$ et $\ell < k''_0$,
contradiction avec (I2).
Finalement, il n'y a bien que des boursiers dans la suite
$d''_k,\ldots,d''_{k''_0-1}$

Montrons $CB((d'''_i)_{i\in 1\ldots |C|},\ell)$
pour $\ell \in k+1\ldots k_0-1$.
Comme montré plus haut,
l'ensemble de candidats $d'''_1,\ldots,d'''_{k}$ contient au moins $q_B  * (k+1)$  candidats boursiers
donc l'ensemble de candidats $d'''_1,\ldots,d'''_{\ell}$ contient au moins 
$q_B  * (k+1) + (\ell - k)$  candidats boursiers et on conclut puisque
$q_B  * (k+1) + (\ell - k) \geq q_B  * (k+1 + \ell - k) = q_B  * (1 + \ell)$

\medskip

On termine en montrant que
 $CB((d'''_i)_{i\in 1\ldots |C|},\ell)$
pour $\ell \in k_0\ldots |C|$.
Puisque $(d''_i)_{i\in 1\ldots |C|}$ satisfait (P1)
alors $CB((d''_i)_{i\in 1\ldots |C|},\ell)$ est vraie.
Les ensembles de candidats $d''_1,\ldots,d''_{\ell}$ et $d'''_1,\ldots,d'''_{\ell}$
coincident à permutation près, donc ils contiennent le même nombre de boursiers
et de plus si il n'y a aucun boursier dans $d''_{\ell+1},\ldots,d''_{|C|}$
alors il n'y en a aucun dans $d'''_{\ell+1},\ldots,d'''_{|C|}$.
Donc $CB((d'''_i)_{i\in 1\ldots |C|},\ell)$ est bien vérifiée.

\end{itemize}



 
  
 
\end{document}